\section{Génération du graphe}\label{sec:graphe}
	Le graphe récupéré par par OSMnx est sous la forme d'un multigraphe orienté : pour tout couple de sommets $(u,v)\in E$ le nombre de liens allant de $u$ à $v$ n'est pas limité à 1. Cela est notamment dû à des rues parallèles.
	
	Afin de faciliter la maniplation du graphe, nous avons créé une classe $CityGraph$ dont le constructeur prends comme paramètre le nom d'une ville et propose différentes méthodes le manipuler ou récupérer des informations.\todo{rm le paragraphe ?}
	% Présenter CityGraph ?

\subsection{Transformation du graphe en Matrice creuse}
	Le graphe est transformé en une matrice creuse au moment de l'écriture des fichiers pour le programme C\todo{expliquer sparse matrix}. Certaines données nécessaires aux calculs peuvent êtres manquantes et sont remplacées par des valeurs par défaut en constantes qui peuvent êtres modifiées
	
	\todo{expliquer algo}
\begin{algorithm}
\caption{Calcul de $V$}
\begin{algorithmic}
	\Require{$G :$ multi-graphe$, ROW\_INDEX, COL\_INDEX$}
	\For{$i\in[\![0,G.NumNodes]\!]$}
		\For{$j$ in $ROW\_INDEX[i,i+1]$}
			\State $weigth \gets []$
			\ForAll{$e$ in $G.edges[i][COL\_INDEX[j]]$}
				\State $weigth.append\left(\frac{e.length}{e.maxspeed}\right)$
			\EndFor
			\State $V[j]\gets min(weigth)$
		\EndFor
	\EndFor
\end{algorithmic}
\end{algorithm}