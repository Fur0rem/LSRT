% -------- %
% Packages %
% -------- %

%% Basics
\usepackage[french]{babel} % langue
\usepackage[utf8]{inputenc} % means input encoding. A voir avec l'encodage des caractères. faut pas chercher à comprendre.
\usepackage[T1]{fontenc} % better accents, and unicode caracters
\usepackage{lmodern} % pareil

%% figures
\usepackage{graphicx} % Required for inserting images
\usepackage{subcaption} % Required for creating figures with multiple parts (subfigures)
\usepackage{float} % Allows putting an [H] in \begin{figure} to specify the exact location of the figure
\usepackage{wrapfig} % Allows in-line images such as the example fish picture
\usepackage{adjustbox} % Required for fitting table to page
\graphicspath{{./images/}}		%chemin vers les images

%% tables
\usepackage[table]{xcolor} % Required for specifying colors by name. table allows to color in tables.
\usepackage{booktabs} % Required for better table rules : \toprule, \midrule, \bottomrule and set the width of the columns
\usepackage{multirow} % Required for multirows in tables
\usepackage{tikz} % Required for drawing custom shapes

%% Math
\usepackage{amsmath} % Required for some math elements
\usepackage{amsthm} % to write theorems, definitions, lemmas, ...
\usepackage{amssymb} % more math symbols

%% mise en page
\usepackage{geometry} % Required for adjusting page dimensions and margins. Adapted the margin for comments
\usepackage{hyperref} % manage hyperlinks
\usepackage{xspace} % space if no punctiation
\usepackage{xcolor} % Required for specifying colors by name
\definecolor{darkgreen}{rgb}{0.0, 0.5, 0.0}
\geometry{margin=20mm} %enough space for margin notes
\hypersetup{colorlinks=true, linkcolor=darkgreen, urlcolor=magenta, citecolor=blue}

% % ------------------- %
% % Writing pseudocodes %
% % ------------------- %
\usepackage{algorithm} % write pseudocode, suggested by IEEEtran instead of algorithm2e
\usepackage{algpseudocode} 

%% TODO notes
\usepackage{todonotes} % Required for the boxes that can be used to write comments, and to-do lists

%% Subsubsubsection
\usepackage{titlesec}

\titleclass{\subsubsubsection}{straight}[\subsection]

\newcounter{subsubsubsection}[subsubsection]
\renewcommand\thesubsubsubsection{\thesubsubsection.\arabic{subsubsubsection}}
\titleformat{\subsubsubsection}
  {\normalfont\normalsize\bfseries}{\thesubsubsubsection}{1em}{}
\titlespacing*{\subsubsubsection}
{0pt}{3.25ex plus 1ex minus .2ex}{1.5ex plus .2ex}

\makeatletter
\def\toclevel@subsubsubsection{4}
\def\l@subsubsubsection{\@dottedtocline{4}{7.0em}{4.1em}}
\makeatother

\setcounter{secnumdepth}{4}
\setcounter{tocdepth}{4}