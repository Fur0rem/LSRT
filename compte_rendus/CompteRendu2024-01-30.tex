
\documentclass[12pt]{article}

\usepackage[utf8]{inputenc}

\title{Compte Rendu reunion du 30/01}
\author{Ivan}

\begin{document}

\maketitle

\section{Compte rendu}

Points abordés : 

\subsection{1 : Clarification mesure}
La notion de "distance" a été clarifiée. Il sera 
mesuré ici le temps pour accéder à un noeud b depuis 
un noeud a au temps t. 

ie : (t, a, b)

Cette mesure est différente des mesures usuelles de 
plus court chemin sur les linkstreams. Elle ressemble 
cependant à la notion de "reachability". 

 

\subsection{2 : Choix d'algorithme}

On utilise Floyd-Warshall, un algorithme basé sur 
Bellman-Ford car la suppression / l'ajout de liens sur 
un link-stream "casse" l'hypothèse de Dijkstra (un nombre 
de liens plus grand implique un trajet plus couteux). 

On utilise pas de parcrours en largeur temporel car cet
algorithme est "incompatible" avec le choix de représentation 
de linkstream. 


\subsection{3 : Clarification sur les attaques} 

On va effectuer plusieurs types d'attaque (aléatoire uniforme, 
centré sur un certain temps, basé sur la betweeness centrality, 
...) . Afin de pouvoir comparer leur efficacité.

Une fois ces mesures effectuées, on pourra effectuer des 
attaques se basant sur des métriques plus complexes (coupes 
de graphes, attaque avec coût,...). 

Il faudra effectuer chaque type d'attaque avec 
différentes "intensités" (différent nombre de liens supprimés) 
pour évaluer leur efficacité.

\subsection{4 : Divers } 

On choisis le graphe Drive d'osmnx pour générer la ville. 

Pour la semaine prochaine, on finira l'implémentation C des 
linkstream, la génération des attaques, la génération des 
graphes. On fusionnera également les branches divergentes.   

\end{document}
