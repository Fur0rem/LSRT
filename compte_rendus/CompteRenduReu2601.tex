\documentclass{article}

\usepackage[utf8]{inputenc}

\title{Compte rendu de la réunion du 26 Janvier}
\date{}

\begin{document}

\maketitle

\section*{Organisation des futures réunions}
\begin{itemize}
	\item Quelqu'un rappelle l'ordre du jour
	\item Chacun dit ce qu'il a fait depuis la dernière fois + questions
	\item Discussion des sujets de la réunion
\end{itemize}

\section*{Sur le code}
\begin{itemize}
	\item Pas de norme spécifique
	\item Valeurs renvoyées passées par paramètre, retour utilisé pour renvoyer un code d'erreur (ou pas d'erreur)
	\item Code lisible (pas de code sur une ligne ou autre)
	\item Optimisation/vitesse de calcule n'est pas l'objectif (inutile de passer 80h à optimiser pour gagner 2 minutes de calcul)
\end{itemize}

\section*{Sur les graphes}
\begin{itemize}
	\item Récupérés avec OSMnx
	\item Orientés (à cause des sens interdits)
	\item Faire attention aux rues parallèles
	\item On préfère avoir plus d'information pour l'implémentation \\
	On a envisagé des matrices creuses (facilite le parallélisme) mais elles prennent de la place pour rien
\end{itemize}

\section*{Distance temporelle}
\begin{itemize}
	\item La littérature s'intéresse au plus court chemin entre deux nœuds dans l'entièreté de l'existence du graphe \\
	cf Présentation de Clémence
	\item On veut tout les chemins
	\item La definition de distance n'est pas sure (arriver en avance rajoute-il du temps) \\
	Les deux sont pertinents et on peu essayer l'une ou l'autre ou les deux
	\begin{itemize}
		\item[$\rightarrow$] Peut être essayer avec l'attente (ie : attendre le temps nécessaire quand on arrive en "avance")
	\end{itemize}
	\item Chemins de l'ordre de la centaine ou du millier de liens
\end{itemize}

\section*{Algos de parcours}
\begin{itemize}
	\item Plusieurs algorithmes de plus court chemin
	\item Algos de Clémence peu adaptés à notre cas (peu de liens dans le sien)
	\item Dans notre cas, plusieurs situations se répètent
	\item Idée d'algorithme : Floyd–Warshall/Bellman–Ford
	\begin{itemize}
		\item Programmation dynamique (conserve les résultats pour ne pas à avoir à les recalculer)
	\end{itemize}
	\item Problèmes
	\begin{itemize}
		\item N'utilise pas les hypothèses que l'on a (ce que l'on sait sur $L$) \\
		Les trajets de chacun des couples de nœuds (bien qu'elles doivent avoir été calculées au préalable) \\
		Si aucun lien entre le départ et l'arrivée ne change, le trajet ne change pas.
		\item Regarde entre deux composantes connexes (qui n'ont donc pas de trajet de l'une vers l'autre)
	\end{itemize}
\end{itemize}

\section*{Algos d'attaque}
\begin{itemize}
	\item Sur papier, bloquer statiquement un endroit serait intéressant, mais on est obligé de bouger dans la vraie vie, d'où l'intéret de penser à des attaques dynamiques
	\item Les algos d'attaque n'existent pas : il faut se baser sur des algos non temporels %%% ?
	\begin{itemize}
		\item retirer en se basant sur le nombre de degrés : Ne marche pas sur une ville
		\item retirer par betweeness centrality réitérée : C'est un algorithme déterministe, il donne toujours la même réponse
	\end{itemize}
	\item Ce qu'il faudrait faire: On se donne un "budget" (nombre de liens à retirer) et créer un algo qui retire différent lieux avec ce budget
	\item Plusieurs options possibles pour générer des attaques à comparer  :
	\begin{itemize}
		\item Retirer des liens en suivant "une trajectoire" \\
	  	ie retirer les voisins a chaque fois
	  	\item Retirer au hasard (benchmark)
	  	\item Retirer "statiquement" des nœuds par betweeness centrality (mesure de bencharck)
	  	\item Avec un certain cout : comparer avec des vraies attaques 
	  	\begin{itemize}
	  		\item À partir d'un trajet connu (vraie manifestation)
	  		\item Quels liens retirés pendant une manifestation ?  
	  	\end{itemize}
	\end{itemize}
\end{itemize}

\section*{Autres remarques}
\begin{itemize}
	\item Pas de GPU sur les serveurs de l'équipe \\
	Plutôt sur les serveurs de calcul
	\item Retirer le bon nombre de liens et regarder l'évolution de la taille de la plus grande composante connexe
	\begin{itemize}
		\item Trop : Casse la ville
		\item Pas assez : Fait un plateau (change peu)
	\end{itemize}
\end{itemize}

\end{document}
